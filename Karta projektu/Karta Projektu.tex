\documentclass[11pt,a4paper]{article}

\usepackage{polski}
\usepackage[utf8]{inputenc}
\usepackage[T1]{fontenc}
\usepackage[polish]{babel}
\usepackage{array}
\usepackage{geometry} 
\usepackage{hyperref}

\newcommand{\Mod}[1]{\ (\mathrm{mod}\ #1)}
\newgeometry{tmargin=3cm, bmargin=3cm, lmargin=0.5cm, rmargin=0.5cm} 

\begin{document}

\begin{center}
    \Large
    Karta Tematu Projektu
\end{center}
\begin{center}
\begin{table}[h]
    \centering
    \begin{tabular}{|p{3cm}|p{5cm}|p{5cm}|}
         \hline
         Grupa 6 & Dziedzina & Data Utworzenia Dokumentu \\ \hline
        \multicolumn{2}{|c|}{Przetwarzanie języka naturalnego} & 09.03.2020 r. \\ \hline
         \multicolumn{2}{|c|}{Temat Projektu} & Data Modyfikacji Dokumentu \\ \hline
         \multicolumn{2}{|c|}{\textbf{Analiza wydźwięku wypowiedzi}} & 09.03.2020 r. \\ \hline
         \multicolumn{3}{|c|}{Członkowie Grupy} \\ \hline
         Lp. & Imię i Nazwisko & Numer Indeksu \\ \hline     
         1 & Miłosz Dziurzyński & 136234 \\ \hline
         2 & Wojciech Gawiński & 136238 \\ \hline
         3 & Jarosław Kmiotek & 135838 \\ \hline
         4 & Rafał Sobański & 136301 \\ \hline
    \end{tabular}
    \label{tab:my_label}
\end{table}
\end{center}

\large
\center
Problemy badawcze

\begin{center}
\end{center}
\begin{table}[h]
    \centering
    \begin{tabular}{|p{4.5cm}|p{9cm}|}
        \hline
         Numer problemu badawczego & 1 \\
         \hline
         Skrócona nazwa problemu & Pozyskanie danych potrzebnych do uczenia maszynowego\\
         \hline
         \multicolumn{2}{|c|}{Cechy zbioru danych} \\ \hline
         1 &  Język angielski \\ \hline
         2 &  Zbiór danych - recenzje z Amazonu\\ \hline
    \end{tabular}
    \label{tab:my_label}
\end{table}

\begin{center}
\end{center}
\begin{table}[h]
    \centering
    \begin{tabular}{|p{4.5cm}|p{9cm}|}
        \hline
         Numer problemu badawczego & 2 \\
         \hline
         Skrócona nazwa problemu & Środowisko uczenia\\
         \hline
         \multicolumn{2}{|c|}{Sposób uczenia} \\ \hline
         1 & Lokalny \\ \hline
         2 &  Uczenie w chmurze (Azure)\\ \hline
    \end{tabular}
    \label{tab:my_label2}
\end{table}


\begin{table}[h]
    \centering
    \begin{tabular}{|p{4.5cm}|p{9cm}|}
        \hline
         Numer problemu badawczego & 3 \\
         \hline
         Skrócona nazwa problemu & Sposób analizy wydźwięku (przykładowe algorytmy)\\
         \hline
         \multicolumn{2}{|c|}{Przetestowane algorytmy} \\ \hline
         1 &   SVM  \\ \hline
         2 &   LDA\\ \hline
         3 &   Word2Vec\\ \hline
         4 &   CNN\\ \hline
         5 &   LSTM\\ \hline 
    \end{tabular}
    \label{tab:my_label3}
\end{table}

\begin{table}[h]
    \centering
    \begin{tabular}{|p{4.5cm}|p{9cm}|}
        \hline
         Numer problemu badawczego & 4 \\
         \hline
         Skrócona nazwa problemu & Interfejs dostępowy do wytrenowanej sieci\\
         \hline
         \multicolumn{2}{|c|}{Wykorzystane technologie do stworzenia API} \\ \hline
         1 &  \\ \hline
         2 &  \\ \hline
         3 &  \\ \hline
         4 &  \\ \hline
    \end{tabular}
    \label{tab:my_label4}
\end{table}

\begin{table}[h]
    \centering
    \begin{tabular}{|p{4.5cm}|p{9cm}|}
        \hline
         Numer problemu badawczego & 5 \\
         \hline
         Skrócona nazwa problemu & Zasada działania autoencodera\\
         \hline
         \multicolumn{2}{|c|}{Wykorzystane dokumenty} \\ \hline
         1 &  \href{http://www.latex-tutorial.com}{OverLeaf}. \\ \hline
         2 &  \href{https://machinelearningmastery.com/lstm-autoencoders/}{LSTM-Autoencoder}\\ \hline
         3 & \href{https://towardsdatascience.com/step-by-step-understanding-lstm-autoencoder-layers-ffab055b6352}{Understanding lstm layers}\\ \hline
    \end{tabular}
    \label{tab:my_label4}
\end{table}

\begin{table}[h]
    \centering
    \begin{tabular}{|p{4.5cm}|p{9cm}|}
        \hline
         Numer problemu badawczego & 4 \\
         \hline
         Skrócona nazwa problemu & Zamiana danych na one-hot vector\\
         \hline
         \multicolumn{2}{|c|}{Wykorzystane materiały} \\ \hline
         1 &  \href{https://scikit-learn.org/stable/}{SkLearn} \\ \hline
         2 &  \href{https://machinelearningmastery.com/how-to-one-hot-encode-sequence-data-in-python/}{Encoding one-hot}\\ \hline
    \end{tabular}
    \label{tab:my_label4}
\end{table}
\end{document}
